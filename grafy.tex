% Options for packages loaded elsewhere
\PassOptionsToPackage{unicode}{hyperref}
\PassOptionsToPackage{hyphens}{url}
%
\documentclass[
]{article}
\usepackage{amsmath,amssymb}
\usepackage{lmodern}
\usepackage{iftex}
\ifPDFTeX
  \usepackage[T1]{fontenc}
  \usepackage[utf8]{inputenc}
  \usepackage{textcomp} % provide euro and other symbols
\else % if luatex or xetex
  \usepackage{unicode-math}
  \defaultfontfeatures{Scale=MatchLowercase}
  \defaultfontfeatures[\rmfamily]{Ligatures=TeX,Scale=1}
\fi
% Use upquote if available, for straight quotes in verbatim environments
\IfFileExists{upquote.sty}{\usepackage{upquote}}{}
\IfFileExists{microtype.sty}{% use microtype if available
  \usepackage[]{microtype}
  \UseMicrotypeSet[protrusion]{basicmath} % disable protrusion for tt fonts
}{}
\makeatletter
\@ifundefined{KOMAClassName}{% if non-KOMA class
  \IfFileExists{parskip.sty}{%
    \usepackage{parskip}
  }{% else
    \setlength{\parindent}{0pt}
    \setlength{\parskip}{6pt plus 2pt minus 1pt}}
}{% if KOMA class
  \KOMAoptions{parskip=half}}
\makeatother
\usepackage{xcolor}
\IfFileExists{xurl.sty}{\usepackage{xurl}}{} % add URL line breaks if available
\IfFileExists{bookmark.sty}{\usepackage{bookmark}}{\usepackage{hyperref}}
\hypersetup{
  pdftitle={Grafy pro seminář KSS/PV1, který se bude konat 4. května 2022},
  pdfauthor={Připravil: František Kalvas},
  hidelinks,
  pdfcreator={LaTeX via pandoc}}
\urlstyle{same} % disable monospaced font for URLs
\usepackage[margin=1in]{geometry}
\usepackage{graphicx}
\makeatletter
\def\maxwidth{\ifdim\Gin@nat@width>\linewidth\linewidth\else\Gin@nat@width\fi}
\def\maxheight{\ifdim\Gin@nat@height>\textheight\textheight\else\Gin@nat@height\fi}
\makeatother
% Scale images if necessary, so that they will not overflow the page
% margins by default, and it is still possible to overwrite the defaults
% using explicit options in \includegraphics[width, height, ...]{}
\setkeys{Gin}{width=\maxwidth,height=\maxheight,keepaspectratio}
% Set default figure placement to htbp
\makeatletter
\def\fps@figure{htbp}
\makeatother
\setlength{\emergencystretch}{3em} % prevent overfull lines
\providecommand{\tightlist}{%
  \setlength{\itemsep}{0pt}\setlength{\parskip}{0pt}}
\setcounter{secnumdepth}{-\maxdimen} % remove section numbering
\ifLuaTeX
  \usepackage{selnolig}  % disable illegal ligatures
\fi

\title{Grafy pro seminář KSS/PV1, který se bude konat 4. května 2022}
\author{Připravil: František Kalvas}
\date{Aktualizováno: 2022-05-04}

\begin{document}
\maketitle

{
\setcounter{tocdepth}{3}
\tableofcontents
}
\hypertarget{uxfavod}{%
\section{Úvod}\label{uxfavod}}

Vážení a milí studující KSS/PV1,

upřímně Vám děkuji za všechna zaslaná data! Níže najdete grafy, které
vznikly z dat Vašich nebo Vašich spolužáků. (Možná jste na jejich
dotazník odpovídali.) Jsou tu data těch, kteří mi je zvládli poslat do
středy 27. 4. 2022 a nezapomněli k nim připojit otázku, nebo vztah,
který je především zajímá. K takto specifikovaným vztahům jsem připravil
grafy a můžete se na ně všichni podívat a zamyslet se, co nám grafy
říkají o vztazích proměnných, které zobrazují. Rád ještě doplním grafy u
těch skupin, které mi sice poslaly data, ale žádné pořádné otázky nebo
vztahy.

Ve středu se na semináři nejprve podíváme na grafy z této stránky.
Řekneme si, jak by je bylo možné interpretovat, jaké jsou základní
principy interpretace dat. Pokud zbyde čas, podíváme se i na data, která
jsem dostal bez dobrých otázek/vztahů. Případně se můžeme podívat i na
data, která jsem zatím nedostal. Každopádně budeme pracovat tak, že si
budeme promítat výsledky a Vy se budete ve skupinkách radit, jak byste
je interpretovali a společně budeme Vaše postřehy reflektovat a
shrnovat.

S úctou,\\
František

\hypertarget{grafy-01-stigmatizace-duux161evnux11b-nemocnuxfdch}{%
\section{Grafy 01: Stigmatizace duševně
nemocných}\label{grafy-01-stigmatizace-duux161evnux11b-nemocnuxfdch}}

Tým, který pořídil data nám vytipoval tři příčiny a dva následky, celkem
tedy 6 vztahů. Zde však najdete jen 3. Proč? Nu, na otázku, zda je
stigmatizace duševně nemocných důležitý problém odpověděli jen 3
respondenti, že \texttt{Ne}, 149 odpovědělo, že \texttt{Ano}, a taková
malá diverzita je při analýze nepoužitelná. Věcně je to hezká správa o
světě -- lidem to není jedno, uvědomují si, že stigmatizace je důležitý
problém. Ale výzkum na tom nepostavíme -- pokud (téměř) všichni
odpovídají \texttt{Ano}, nedá se určit příčina. Budeme se tedy nakonec
zabývat tím, co ovlivňuje názor, že stigmatizace je způsobená
neinformovaností.

\hypertarget{vliv-informovanosti}{%
\subsection{Vliv informovanosti}\label{vliv-informovanosti}}

Nejprve se podíváme na vliv informovanosti. Proměnná je měřená jako
subjektivní dojem respondenta, jak je informovaný o problematice
duševního onemocnění.

\includegraphics{grafy_files/figure-latex/graf 2.1-1.pdf}

\hypertarget{vliv-vzdux11bluxe1nuxed}{%
\subsection{Vliv vzdělání}\label{vliv-vzdux11bluxe1nuxed}}

Nyní se podíváme na vliv vzdělání. Proměnná je měřená jako nejvyšší
dosažené vzdělání. Aby se výsledky netříštily spoustou kategorií,
sloučil jsem různé stupně VŠ vzdělání do kategorie \texttt{VŠ} a vyšší
odborné a střední s maturitou jsem sloučil do \texttt{SŠ}.

\includegraphics{grafy_files/figure-latex/graf 2.2-1.pdf}

\hypertarget{vliv-svux11bdectvuxed}{%
\subsection{Vliv svědectví}\label{vliv-svux11bdectvuxed}}

Nakonec se podíváme na vliv skutečnosti, zda byl respondent svědkem
stigmatizace lidí s duševním onemocněním.

\includegraphics{grafy_files/figure-latex/graf 2.3-1.pdf}

\hypertarget{grafy-02-adopce-dux11btuxed}{%
\section{Grafy 02: Adopce dětí}\label{grafy-02-adopce-dux11btuxed}}

Tým, který pořídil data nám vytipoval jednoduchý vztah: Ovlivňuje věk
respondentů ochotu adoptovat si dítě? Věk je kategorizovaný, přičemž
kategorie \texttt{10-15} a \texttt{51\ a\ více} se každá vyskytují jen
jednou, proto je připojím k nejbližším kategoriím, aby se výsledky
nedrobily. Jako další možné zajímavé faktory jsem narychlo vytipoval
pohlaví a fakt, zda má respondent již děti. V datech jsem nechal ještě
druhou závisle proměnnou: zda by respondent adoptoval postižené dítě.

\hypertarget{samotnuxe1-adopce-vux11bk-a-pohlavuxed}{%
\subsection{Samotná adopce: Věk a
pohlaví}\label{samotnuxe1-adopce-vux11bk-a-pohlavuxed}}

Ovlivňuje ochotu k prosté adopci Věk v souhře s pohlavím?

\includegraphics{grafy_files/figure-latex/graf 1.1-1.pdf}

\hypertarget{samotnuxe1-adopce-vux11bk-a-rodiux10dovstvuxed}{%
\subsection{Samotná adopce: Věk a
rodičovství}\label{samotnuxe1-adopce-vux11bk-a-rodiux10dovstvuxed}}

Máme jen 40 respondentů, to je určitě málo na to udělat vliv celé
trojkombinace věku, pohlaví a rodičovství. Podívejme se tedy zvlášť na
vliv kombinace rodičovství a věku. Ovlivňuje ochotu k prosté adopci věk
v souhře s rodičovstvím? Pozor! Není to chyba, že některé sloupečky
chybí! To znamená, že pro některé kombinace nejsou pozorování\ldots{} Co
jsme se díky této `náhodě' dozvěděli?

\includegraphics{grafy_files/figure-latex/graf 1.2-1.pdf}

\hypertarget{samotnuxe1-adopce-rodiux10dovstvuxed-a-pohlavuxed}{%
\subsection{Samotná adopce: Rodičovství a
pohlaví}\label{samotnuxe1-adopce-rodiux10dovstvuxed-a-pohlavuxed}}

Ovlivňuje ochotu k prosté adopci rodičovství v souhře s pohlavím?

\includegraphics{grafy_files/figure-latex/graf 1.3-1.pdf}

\hypertarget{adopce-postiux17eenuxe9ho-vux11bk-a-pohlavuxed}{%
\subsection{Adopce postiženého: Věk a
pohlaví}\label{adopce-postiux17eenuxe9ho-vux11bk-a-pohlavuxed}}

Ovlivňuje ochotu k adopci postiženého dítěte věk v souhře s pohlavím?

\includegraphics{grafy_files/figure-latex/graf 1.4-1.pdf}

\hypertarget{adopce-postiux17eenuxe9ho-vux11bk-a-rodiux10dovstvuxed}{%
\subsection{Adopce postiženého: Věk a
Rodičovství}\label{adopce-postiux17eenuxe9ho-vux11bk-a-rodiux10dovstvuxed}}

Ovlivňuje ochotu k adopci postiženého dítěte věk v souhře s
rodičovstvím?

\includegraphics{grafy_files/figure-latex/graf 1.5-1.pdf}

\hypertarget{adopce-postiux17eenuxe9ho-rodiux10dovstvuxed-a-pohlavuxed}{%
\subsection{Adopce postiženého: Rodičovství a
pohlaví}\label{adopce-postiux17eenuxe9ho-rodiux10dovstvuxed-a-pohlavuxed}}

Ovlivňuje ochotu k adopci postiženého dítěte rodičovství v souhře s
pohlavím?

\includegraphics{grafy_files/figure-latex/graf 1.6-1.pdf}

\hypertarget{rozliux161ovuxe1nuxed-typu-adopce-podle-vux11bku-a-pohlavuxed}{%
\subsection{Rozlišování typu adopce podle věku a
pohlaví}\label{rozliux161ovuxe1nuxed-typu-adopce-podle-vux11bku-a-pohlavuxed}}

\includegraphics{grafy_files/figure-latex/graf 1.7-1.pdf}

\hypertarget{rozliux161ovuxe1nuxed-typu-adopce-podle-rodiux10dovstvuxed-a-pohlavuxed}{%
\subsection{Rozlišování typu adopce podle rodičovství a
pohlaví}\label{rozliux161ovuxe1nuxed-typu-adopce-podle-rodiux10dovstvuxed-a-pohlavuxed}}

\includegraphics{grafy_files/figure-latex/graf 1.8-1.pdf}

\hypertarget{grafy-03-syndrom-vyhoux159enuxed}{%
\section{Grafy 03: Syndrom
vyhoření}\label{grafy-03-syndrom-vyhoux159enuxed}}

Další projekt se věnoval syndromu vyhoření. Kladl si otázku, jestli
respondenti ví, co syndrom vyhoření je, což se ukázalo jako nezajímavá
položka, neboť všichni odpověděli správně. Dále si tým klade otázku, zda
si respondenti myslí, že stát má preventivní opatření. Tak se k tomu
koukneme na graf! Ano, není tam příčina, nevadí, třeba ji najdeme na
hodině:

\includegraphics{grafy_files/figure-latex/graf 3.1-1.pdf}

Z pilnosti jsem vytvořil ještě jeden graf -- jak často respondenti
uváděli jednotlivé aspekty syndromu vyhoření:

\includegraphics{grafy_files/figure-latex/graf 3.2-1.pdf}

\hypertarget{grafy-04-motivace-studovat-sociuxe1lnuxed-pruxe1ce}{%
\section{Grafy 04: Motivace studovat sociální
práce}\label{grafy-04-motivace-studovat-sociuxe1lnuxed-pruxe1ce}}

Další data jsem dostal bez otázek. Upravil jsem je a ukážu Vám tady graf
na impulsy studovat sociální práce a graf na zamýšlné cílové skupiny:

\includegraphics{grafy_files/figure-latex/graf 4.1-1.pdf}

\includegraphics{grafy_files/figure-latex/graf 4.2-1.pdf}

\hypertarget{grafy-05-pux159uxedstup-k-postiux17eenuxfdm-osobuxe1m}{%
\section{Grafy 05: Přístup k postiženým
osobám}\label{grafy-05-pux159uxedstup-k-postiux17eenuxfdm-osobuxe1m}}

Poslední data opět přišla bez otázky. Vyčistil jsem je a zkusil vyrobit
nějaký graf -- jaké jsou reakce respondentů na tělesné a mentální
postižení:

\includegraphics{grafy_files/figure-latex/graf 5.1-1.pdf}

\end{document}
